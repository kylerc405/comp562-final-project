\documentclass[11pt]{article}

\usepackage[margin=1in]{geometry}
\usepackage{graphicx}
\usepackage{times}
\usepackage{url}

\title{Predicting and Recommending Skills for Job Candidates}
\author{Alex Tang \\
Eric Ge \\ 
Guru Balamurugan \\ 
Kyler Chen \\
}

\date{}

\begin{document}

\maketitle

\section{Introduction}
In an increasingly competitive job market, people seeking employment often face a complex landscape of required skills and qualifications. Employers frequently include a variety of requirements in their job postings, including both domain-specific technical abilities as well as more general interpersonal/analytical skills. Therefore, we see identifying and bridging the gap between a candidate's current skill set and the ideal capabilities required for a specific role as an important challenge. This report investigates the feasibility of utilizing machine learning methods to predict a job candidate's level of qualification as well as providing suggestions on which skills they may be missing. 

While recruiters, hiring managers, and workforce development agencies already use systems like this to understand the talent landscape better and identify potential candidates, our aim is for this technology to aid applicants themselves directly, by allowing them to identify areas of improvement within their academic and professional journeys. Furthermore, educational and training institutions can benefit from this as well, through tailoring their academic curriculum based on skill gaps observed in the market. The broad goal is to improve the career application processes of students by making skill requirements more transparent and actionable. 

This project idea draws on existing work in skill extraction, job recommendation systems, and text mining of job descriptions. Although no external collaborator or professor specifically contributed to this particular work, we have been informed and inspired by prior research and open-source frameworks. The full code for this project, once completed, will be accessible at: \url{https://github.com/kylerc405/comp562-final-project}.

\section{Problem and Motivation}
The problem addressed in this project is to determine the missing skills a candidate should acquire to align themselves more closely with a given job opportunity. Unlike simpler classification tasks, this problem requires considering multiple sets of attributes: the candidate’s current skill set, the job’s commonly-seen skill set, and the description of the job. The output is not just a binary decision (e.g., suitable or not) but rather a ranked list of recommended skills for the candidate to learn next, as well as a numeric score to better judge suitability. 

The motivation for this project stems from the growing complexity of the job market and the mismatch often seen between candidate skills and job requirements. When a job seeker begins applying for roles, we believe they likely would benefit from a straightforward, consolidated list of the skills/technologies they would benefit most from learning, based on real-world job data. Thus, they are given the opportunity to spend their time on the most valuable tasks for achieving their desired job field. Candidates who can identify and target specific missing skills stand a better chance of securing employment or progressing in their careers. The economic impact could be significant: improving candidate-job matching can reduce underemployment, support career transitions, and guide individuals toward more purposeful skill development.
\section{Application Domain}
The envisioned application involves both individual candidates and organizations that provide career guidance. For an individual user, the system would function as a personalized recommender: given their existing skill set and a target job’s requirements, the model would output a set of skills that, if learned, could increase their competitiveness for the role, as well as a score depicting their current level of desirability based on their resume. Such a tool could prove beneficial in improving their resume's efficacy in describing their abilities, as well as informing an applicant on whether or not a given job is worth their time to consider applying for. For organizations, such a tool could be integrated into career counseling platforms, learning management systems, or job portals, providing automated suggestions for skill enhancement pathways.

An example scenario might be a mid-career professional aiming to transition from a general administrative position into a business development role. Using their current known skills (e.g., ``Microsoft Excel'', ``Customer Service'', ``Scheduling''), the model examines the target position’s description (e.g., requiring ``Sales Strategy'', ``CRM Software'', ``B2B Relationship Management'') and identifies missing competencies (e.g., familiarity with ``Salesforce CRM''). The candidate could then focus their efforts on developing these targeted skills, either through online courses, workshops, or formal training, rather than aimlessly upskilling in random areas.

\section{Related Work}
TODO (see References for some examples)
\section{Approach and Methodology}

\subsection{Data Loading and Preprocessing}

The dataset used for this analysis was sourced from \url{https://www.kaggle.com/datasets/suriyaganesh/resume-dataset-structured?resource=download}, containing structured information extracted from professional resumes, normalized into multiple related tables. The primary goal was to perform a skills gap analysis using machine learning, aiming to predict the likelihood of an individual securing a computer science (CS)-related role based on their skills. This involved:

\begin{itemize}
    \item Identifying key skills that distinguish successful CS professionals from others.
    \item Using model-derived feature importance to highlight skills gaps.
\end{itemize}

\textbf{Preprocessing Steps:}
\begin{itemize}
    \item \textit{Data Cleaning:} Imputation and normalization were applied to cleanse the data of NaNs/nulls and lexical inconsistencies such as commas and capitalization.
    \item \textit{Date Processing:} Unstructured date formats were standardized to facilitate processing, ensuring that only the most recent work experiences and associated skills were considered for each resume.
\end{itemize}

\subsection{Skill and Job Title Mapping}

\textbf{Mapping Strategy:}
\begin{itemize}
    \item Job titles were mapped against a predefined list of titles commonly associated with the CS field. Titles that matched were classified as CS professional roles.
\end{itemize}

\subsection{Regression and Target Variable Definition}

Instead of a binary classification of CS vs. Non-CS based on job titles, a continuous scoring mechanism was implemented:
\begin{itemize}
    \item A score ranging from 0 to 100 was computed to quantify the aptitude of a candidate as a CS professional, factoring in weighted skills and educational qualifications such as degrees (e.g., Masters, Bachelors).
    \item This composite score was then used as a continuous target variable for the model, allowing for the ranking of individuals relative to one another based on their qualifications and skills.
\end{itemize}

\subsection{Model Building and Processing}

\textbf{Feature Encoding and Dimensionality Reduction:}
\begin{itemize}
    \item Due to the presence of over 200,000 unique skills in the dataset, frequency filtering was applied to focus on skills that appeared more than 100 times in relation to CS-specific entries.
    \item Skills were further processed using a MultiLabelBinarizer for vectorization and Truncated SVD for dimensionality reduction, simplifying the feature space to the most relevant and common skills within the CS profession.
\end{itemize}

\textbf{Model Training:}
\begin{itemize}
    \item A Random Forest classifier was trained using the processed features to predict the likelihood of candidates securing CS roles based on their resume data.
\end{itemize}

\section{Evaluation and Impact}
TODO (see last 2 cells)

\section{Conclusion and Future Work}
This project proposes a machine learning-based system to analyze a candidate’s skill set in relation to a target job’s requirements and recommend additional skills for the candidate to acquire. The motivation lies in improving job-market alignment, aiding candidates in strategic upskilling, and ultimately contributing to a more efficient and equitable hiring process.

This report focuses on the conceptual foundations, application domain and use cases, programmatic implementations, and proposes future work such as an improved language processing and further considering ethical and bias-related implications, ensuring that the system does not unintentionally disadvantage certain candidate groups. The project code and experimental artifacts will be shared via our public GitHub repository: \url{https://github.com/kylerc405/comp562-final-project }.

TODO: IMPROVE FUTURE WORK SECTION maybe
% References
\newpage
\section*{References}
\url{https://github.com/giterdun345/Job-Description-Skills-Extractor}
\\
\\
\url{https://github.com/Deba951/Resume-ATS-Tracking-LLM-Project}
\\
\\
\url{https://github.com/deepakpadhi986/AI-Resume-Analyzer}

\end{document}
